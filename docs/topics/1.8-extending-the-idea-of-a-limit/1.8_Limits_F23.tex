\documentclass[12pt]{article}
\usepackage{graphicx}
\usepackage{amssymb}
\usepackage{epstopdf}
\usepackage{amsmath}
\usepackage{multicol}
\usepackage{tcolorbox}
\usepackage{geometry}
\usepackage{enumitem}
\usepackage{fancyhdr}

\DeclareGraphicsRule{.tif}{png}{.png}{`convert #1 `dirname #1`/`basename #1 .tif`.png}

\textwidth = 6.5 in
\textheight = 9 in
\oddsidemargin = 0.0 in
\evensidemargin = 0.0 in
\topmargin = -23pt
\headheight = 0.0 in
\headsep = 0.0 in
\parskip = 0.2in
\parindent = 0.0in
\pagestyle{fancy}
\pagenumbering{gobble}

\newtheorem{theorem}{Theorem}
\newtheorem{corollary}[theorem]{Corollary}
\newtheorem{definition}{Definition}
%\includegraphics [height=50mm, width=50mm]{PathInt.jpg}
\title{Title} 

\begin{document}
%INSTRUCTOR NOTES
%No warm-up on quiz days. If wanted, today's warm-up would be solving equations.
 Name:
 \begin{center}\large{1.8 Limits and Continuity}\end{center}

\begin{enumerate}


\item  \textbf{Holes, Asymptotes, and Indeterminate forms:} Compare the following functions. For each, Find the limit as $x$ approaches 2, and determine whether the function is continuous at $x=2$.
	\begin{multicols}{2}
	\begin{enumerate}[itemsep=1cm]
	\item $\displaystyle f(x)=3x+5$
	\item $\displaystyle g(x)=\frac{(3x+5)(x-2)}{x-2}$
	\item $\displaystyle h(x)=\frac{3x+5}{x-2}$
	\item $\displaystyle g(x)=\frac{(3x+5)(x-2)}{(x-2)^2}$
	\end{enumerate}
	\end{multicols}
\vfill
\begin{tcolorbox}
\textbf{Summary:}\\
If a function is of the form $\displaystyle f(a)=\frac{K}{0}$ where $K$ is a constant, then $\displaystyle \lim_{x\to a} f(x) $....\\
\vspace{10mm}\\
If a function is of the form $\displaystyle f(a)=\frac{0}{0}$, then $\displaystyle \lim_{x\to a} f(x) $....\\
\vspace{10mm}
\end{tcolorbox}


\item \textbf{Limits at Infinity:} 
Use graphs to find the limits at infinity for the following functions:
	\begin{multicols}{2}
	\begin{enumerate}[itemsep=1cm]
	\item $\displaystyle \lim_{x\to -\infty} e^{-x}$
	\item $\displaystyle \lim_{x\to \infty}\frac{1}{x^2}+2$
	\item $\displaystyle \lim_{x\to \infty}\frac{4x^2-8}{-7x^2}$
	\item $\displaystyle \lim_{x\to -\infty}\frac{3x^3-27}{x-3}$
	\item $\displaystyle \lim_{x\to -\infty}\frac{3\sqrt{x}-2}{x^5+4}$
	\item $\displaystyle \lim_{x\to \infty}\sin(x)$
	
	\end{enumerate}
	\end{multicols}
\newpage

\hspace{20px}
\begin{tcolorbox}
\textbf{Summary:}
For rational function of the form $r(x)=\frac{p(x)}{q(x)}$ with $p(x)$ degree $n$ and $q(x)$ degree $m$, we have:\\
\begin{itemize}
\item If $n<m$...\\

\item  If $n>m$...\\
\item  If $n=m$...\\
\vspace{15mm}
\end{itemize}
\end{tcolorbox}

\item Find the value of $k$ that would make the function continuous.

	\[ g(x) = \begin{cases} 
      \frac{e^x-1}{x}& x\neq 0 \\
      k & x = 0 \\
   \end{cases} \]
\vfill

\item Find a value of $m$ that would make the limit exist. Find the limit.
$$ \lim_{x\to\infty}\frac{2x^3-6}{x^m+3}$$

\vfill
\item For each description, sketch a graph with the given characteristics:
	\begin{enumerate}
	\item $f(4)$ is undefined and $\displaystyle \lim_{x\to4} f(x)=2$
	\vfill
	\item $g(3)=2$ and $\displaystyle \lim_{x\to3} g(x)$ does not exist
	\vfill
	\item $h(4) = 2$,  $\displaystyle \lim_{x\to4^-} h(x)=0$, and $\displaystyle \lim_{x\to4^+} h(x)=-\infty$
	\vfill
	\end{enumerate}
	
\item BONUS: The Dirichlet Function is defined as 
$$f(x) = \left\{ \begin{array}{ll} x \quad \text{if } x \text{ is rational }  \\ 0 \quad \text{otherwise} \end{array} \right. $$
Is this function continuous at $x=0$?
\vfill
\end{enumerate}
\end{document} 

%%%%%%%%%

\item Find each limit. Include a table of values to illustrate your answer. \\

	\begin{multicols}{2}
	\begin{enumerate}[itemsep=3cm]
	\item $\displaystyle \lim_{x\to 0} \left(1+x\right)^{\frac{1}{x}}=$
	\item $\displaystyle \lim_{\theta \to 0} \frac{\sin(2\theta)}{\theta}=$
	\item $\displaystyle \lim_{y \to \infty} \frac{\sqrt{y^2+2}}{5y-6}=$
	\item $\displaystyle \lim_{t \to 1^+} \frac{|1-t|}{1-t}=$
\end{enumerate}
\end{multicols}
\begin{tcolorbox} 
\textbf{Definitions:}

\begin{itemize} 
\item A function $f$ is continuous at $c$ if $\displaystyle \lim_{x \to c} f(c) = f(c)$. In other words...
\vspace{15mm}

\item The limit $L$ of a function $f$ at $c$ exists if $f$ is defined near $c$ (except possibly at $c$ itself), and $\displaystyle \lim_{x\to c^+} f(x) = \lim_{x\to c^-} f(x)$. In other words...
\vspace{15mm}

%\item \textbf{Intermediate Value Theorem:} Suppose $f$ is continuous on a closed interval $[a,b]$.\\ If $k$ is any number between $f(a)$ and $f(b)$, then there exists at least one number\\ $c$ in $[a,b]$ such that $f(c)=k$.

\end{itemize}
\end{tcolorbox}




\begin{tcolorbox}
\textbf{Warm-up: } Solve the following equations for $t$.
\begin{multicols}{2}
\begin{enumerate}
\item $(t+1)^2=9$
\item $tx+x^2=5$
\end{enumerate}
\end{multicols}
\end{tcolorbox}

MINIPAGE
\noindent\begin{minipage}{0.3\textwidth}% adapt widths of minipages to your needs
try 1
\end{minipage}%
\hspace{40mm}
\begin{minipage}{0.6\textwidth}
a) $f'(2)=$\\\

b) $f'(4)=$\\

c) $f'(6)=$\\

d) $f'(7)=$\\

e) $f'(8)=$
\end{minipage}
