\documentclass[12pt]{article}
\usepackage{graphicx}
\usepackage{amssymb}
\usepackage{epstopdf}
\usepackage{amsmath}
\usepackage{multicol}
\usepackage{tcolorbox}
\usepackage{geometry}
\usepackage{enumitem}
\usepackage{fancyhdr}

\DeclareGraphicsRule{.tif}{png}{.png}{`convert #1 `dirname #1`/`basename #1 .tif`.png}

\textwidth = 6.5 in
\textheight = 9 in
\oddsidemargin = 0.0 in
\evensidemargin = 0.0 in
\topmargin = -23pt
\headheight = 0.0 in
\headsep = 0.0 in
\parskip = 0.2in
\parindent = 0.0in
\pagestyle{fancy}
\pagenumbering{gobble}

\newtheorem{theorem}{Theorem}
\newtheorem{corollary}[theorem]{Corollary}
\newtheorem{definition}{Definition}
%\includegraphics [height=50mm, width=50mm]{PathInt.jpg}
\title{Title} 

\begin{document}
%INSTRUCTOR NOTES
%This activity aligns with a Desmos activity for exploring families of functions.

 Name:
 \begin{center}\large{4.7 L'Hopital's Rule}\end{center}


\begin{tcolorbox}	
\textbf{Theorem: L'Hopital's Rule}\\
Suppose $\displaystyle \lim_{x\to a} \frac{f(x)}{g(x)}$ is given where $f(x)$ and $g(x)$ are differentiable near $a$ and \\
$$\displaystyle \lim_{x\to a} f(x)= \lim_{x\to a} g(x)=0$$
 $$OR$$
  $$\displaystyle \lim_{x\to a} f(x)= \lim_{x\to a} g(x)=\pm \infty.$$
  
  Then \\
  \end{tcolorbox}
 \begin{tcolorbox} 
  \textbf{Indeterminate Forms}\\
When evaluating a limit results in one of the following forms, \\there is not yet enough information to evaluate the limit.

$\displaystyle \frac{0}{0}$ \hspace{5mm} 
$\displaystyle \frac{\infty}{\infty}$ \hspace{5mm}
$\displaystyle \frac{-\infty}{-\infty}$ \hspace{5mm}
$\displaystyle \infty-\infty$ \hspace{5mm}
$\displaystyle \infty^0$ \hspace{5mm}
$\displaystyle 1^{\infty}$ \hspace{5mm}
$\displaystyle 0^0$ \hspace{5mm}
	\vspace{10mm}
  \end{tcolorbox}
  \newpage
 
$\hspace{10px}$ \\

Evaluate the following limits. Be sure to first determine whether L'Hopital's rule applies.

	\begin{enumerate}
		
	\item $\displaystyle \lim_{x\to0} \frac{\sin(2x)}{x}$
	\vfill
	
	\item $\displaystyle \lim_{x\to \infty} x^{-2}e^{x}$
	\vfill
		\item $\displaystyle \lim_{x\to \infty } \frac{e^{-x}}{\sin x} $
	\vfill
	
	 \item $\displaystyle \lim_{x\to 1 } \frac{\ln x}{x^2-1} $
	\vfill
	\item $\displaystyle \lim_{x\to0^+}x\ln x$
		\vfill
	\item $\displaystyle \lim_{x\to0}\frac{x+\cos x}{x}$
		\vfill
	\item $\displaystyle \lim_{x\to1}\left(1+\sin\left(\frac{3}{x}\right)\right)^x$
		\vfill	
	\end{enumerate}


\end{document}
%%%%%%%%%%%%%%%%%%%%%%%%%%%%%%%%%%%%%%%%%%%
\item \textbf{Potential Energy} \\For positive $a$, $b$, the potential energy, $U$, of a particle moving along the $x$-axis is $\displaystyle U(x)=b\left(\frac{a^2}{x^2}-\frac{a}{x}\right)+b$, where for $x>0$.
		\begin{enumerate}
	\item Find the intercepts and asymptotes. What do they represent in this context?
	\vfill
	\item Compute the local maxima and minima. What do they represent in this context?
	\vfill\vfill\vfill
	\end{enumerate}

\item \textbf{The Quadratic Family} Consider the function $\displaystyle f(x)=a(x-h)^2+k$ where $a$, $h$, and $k$ are parameters.
	\begin{enumerate}
	\item How does changing $a$ impact the shape of the graph?
	\item How does changing $h$ and $k$ impact the shape of the graph?
	\item Use calculus to find the critical points of $f$. (Treat $a$, $h$, and $k$ as constants. Your solution may depend on at least one of $a$, $h$, or $k$.) Is the critical point a max or a min? 
	\item Use calculus to find the inflection points of $f$. (Treat $a$, $h$, and $k$ as constants. Your solution may depend on at least one of $a$, $h$, or $k$.) Explain your results.
	\end{enumerate}
