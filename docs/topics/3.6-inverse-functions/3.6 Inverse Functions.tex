\documentclass[12pt]{article}
\usepackage{graphicx}
\usepackage{amssymb}
\usepackage{epstopdf}
\usepackage{amsmath}
\usepackage{multicol}
\usepackage{tcolorbox}
\usepackage{geometry}
\usepackage{enumitem}
\usepackage{fancyhdr}
\usepackage{pifont,amssymb} % for the symbols

\newlist{todolist}{itemize}{2}
\setlist[todolist]{label=$\square$}


\textwidth = 6.5 in
\textheight = 9 in
\oddsidemargin = 0.0 in
\evensidemargin = 0.0 in
\topmargin = -23pt
\headheight = 0.0 in
\headsep = 0.0 in
\parskip = 0.2in
\parindent = 0.0in
\pagestyle{fancy}
\pagenumbering{gobble}

\newtheorem{theorem}{Theorem}
\newtheorem{corollary}[theorem]{Corollary}
\newtheorem{definition}{Definition}
%\includegraphics [height=50mm, width=50mm]{PathInt.jpg}
\title{Title} 

\begin{document}
%INSTRUCTOR NOTES

 Name:
 \begin{center}\large{3.6 Derivatives of Inverse Functions}\end{center}

\begin{enumerate}
\item The equation $C=\frac{5}{9}(F-32)$ relates a temperature given in F degrees Fahrenheit to the corresponding temperature C measured in degrees Celsius. 
	\begin{enumerate}
	\item Solve the equation for $F$ to write $F$ (Fahrenheit temperature) in terms of $C$ (Celsius temperature).
	\vfill
	\item Using the first equation, compute $\frac{dC}{dF}$ and interpret the units. 
	\vfill
	\item Using the inverse equation you found compute $\frac{dF}{dC}$ and interpret the units. 
	\vfill
	\item What do you notice?
	\vfill
	\end{enumerate}
\begin{tcolorbox}

\textbf{Derivatives of Inverse Functions} \\
$$\frac{d}{dx}\left(f^{-1}(x)\right)=\frac{1}{f'\left(f^{-1}(x)\right)}$$

This means....
\vspace{15mm}
\end{tcolorbox}

\item Suppose a function $f\left(x\right)$ has an inverse given by $f^{-1}\left(x\right)=g\left(x\right)$. We also know that $f\left(3\right)=-1$ and $f'\left(3\right)=10$. Which of the following are true? (You may select more than one option.) 

	  \begin{todolist}
	\item $g(-1)=3$
	\item $g'(3)=\frac{1}{10}$
	\item $g'(-1)=-10$
	\item $g'(-1)=\frac{1}{10}$
	\item $g'(10)=3$
 	 \end{todolist}
\newpage
~
\item Match each function to its derivative function.

\begin{tabular}{l@{\hskip 2.5in}l}
(a) $f(x)=\ln(x)$ & 1. $f'(x)=\frac{1}{\sqrt{1-x^2}}$\\
&\\
(b) $f(x)=\arcsin(x)$ & 2. $f'(x)=\frac{1}{1+x^2}$\\
&\\
(c) $f(x)=\arctan(x)$ & 3. $f'(x)=\frac{1}{x}$\\
\end{tabular}

\item  Compute the derivatives of the following functions. Note the importance of the domain for each.
\begin{enumerate}
	\item $\displaystyle f(x)=\ln(1-e^{-x})$
	\vfill
	\item $\displaystyle g(x)=\cos(\arctan 3x) $
	\vfill
	\end{enumerate}


\item Average leaf width, $w$ (in mm), in tropical Australia is a function of the average annual rainfall, $x$ (in mm). We have $$w=f(x)=32.7 \ln\left(\frac{x}{24.5}\right)$$
	\begin{enumerate}
	\item Find $f'(x)$.
	\vfill
	\item Find $f'(2000)$. Include units.
	\vfill
	\item Explain how you can use your answer to part (b) to estimate the difference in average leaf widths in a forest whose average annual rainfall is 2000 mm and one whose annual rainfall is 150 mm more.
	\vfill
	\end{enumerate}



\end{enumerate}
\end{document} 
A company charges a flat rate of \$500 to rent up to 100 chairs and every additional chair costs \$4. Draw a graph and find a formula for the cost $C$ as a function of $n$, the number of chairs rented.\\
%%%%%%%%%

begin{multicols}{2}
\begin{enumerate}[itemsep=1cm]
\item $\displaystyle \frac{d}{d\theta} \sin(\theta) = $
\item $\displaystyle \frac{d}{d\theta} \cos(\theta) = $
\item $\displaystyle \frac{d}{d\theta} \tan(\theta) = $\\
\item $\displaystyle \frac{d}{d\theta} \csc(\theta) = $
\item $\displaystyle \frac{d}{d\theta} \sec(\theta) = $
\item $\displaystyle \frac{d}{d\theta} \cot(\theta) = $\\
\end{enumerate}
\end{multicols}

\begin{tcolorbox}
\textbf{Warm-up: } Solve the following equations for $t$.
\begin{multicols}{2}
\begin{enumerate}
\item $(t+1)^2=9$
\item $tx+x^2=5$
\end{enumerate}
\end{multicols}
\end{tcolorbox}	

MINIPAGE
\noindent\begin{minipage}{0.3\textwidth}% adapt widths of minipages to your needs
try 1
\end{minipage}%
\hspace{40mm}
\begin{minipage}{0.6\textwidth}
a) $f'(2)=$\\\

b) $f'(4)=$\\

c) $f'(6)=$\\

d) $f'(7)=$\\

e) $f'(8)=$
\end{minipage}
