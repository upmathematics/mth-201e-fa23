\documentclass[12pt]{article}
\usepackage{graphicx}
\usepackage{amssymb}
\usepackage{epstopdf}
\usepackage{amsmath}
\usepackage{multicol}
\usepackage{tcolorbox}
\usepackage{geometry}
\usepackage{enumitem}
\usepackage{fancyhdr}

\DeclareGraphicsRule{.tif}{png}{.png}{`convert #1 `dirname #1`/`basename #1 .tif`.png}

\textwidth = 6.5 in
\textheight = 9 in
\oddsidemargin = 0.0 in
\evensidemargin = 0.0 in
\topmargin = -23pt
\headheight = 0.0 in
\headsep = 0.0 in
\parskip = 0.2in
\parindent = 0.0in
\pagestyle{fancy}
\pagenumbering{gobble}

\newtheorem{theorem}{Theorem}
\newtheorem{corollary}[theorem]{Corollary}
\newtheorem{definition}{Definition}
%\includegraphics [height=50mm, width=50mm]{PathInt.jpg}
\title{Title} 

\begin{document}
%INSTRUCTOR NOTES

 Name:
 \begin{center}\large{2.1 How do we measure speed?}\end{center}




\begin{tcolorbox}
\textbf{Definitions}

The \textbf{average velocity} of an object is the change in position per unit change in time. Over time interval $a\leq t \leq b$, where $s(t)$ is the position of the object at time $t$, it is given by 
	\vspace{10mm}

The \textbf{instantaneous velocity} of an object at a single point in time $t=a$ where position is $s(t)$ is position at time $t$ is 

	
	\vspace{10mm}
\end{tcolorbox}
	
\begin{enumerate}
\item In a time of $t$ seconds, a particle moves a distance of $s$ meters from its starting point, where $s=f(t)=t^2+1$. 
	\begin{enumerate}
	\item Find the average velocity between $t=2$ and $t=2.1$.
	\vfill
	\item Find the average velocity between $t=2$ and $t=2.01$.
	\vfill
	\item Find the average velocity between $t=2$ and $t=2.001$.
	\vfill
	\item Give your best estimate of the instantaneous velocity of the particle at $t=2$.
	\end{enumerate}

\item The position of a car traveling along a straight east/west highway at various times is shown in the table below. Positive values of $d$ indicate that the car is east of its starting point, while negative values of $d$ indicate that the car is west of its starting point.\\

\begin{tabular}{l|lllll} 
time (hours) & 1&2&3&4&5\\
\hline
position (miles) & 40 & -10 & 20 & 90 & -50\\
\end{tabular}

	\begin{enumerate}
	\item Calculate the average velocity of the car between 1 and 2 hours.
	\vfill
	\item Calculate the average velocity of the car between 2 and 4 hours.
	\vfill
	\item What does a negative velocity mean?
	\vfill
	\end{enumerate}


\end{enumerate}
\end{document} 

%%%%%%%%%
Learning Objectives:
\begin{itemize}
\item How do we measure speed?
\item Why is the slope value a measurement of average rate of change?
\item How do we numerically compute average rate of change?
\item What is the difference between average rate of change and instantaneous rate of change?
\end{itemize}
\begin{tcolorbox}
\textbf{Warm-up: } Solve the following equations for $t$.
\begin{multicols}{2}
\begin{enumerate}
\item $(t+1)^2=9$
\item $tx+x^2=5$
\end{enumerate}
\end{multicols}
\end{tcolorbox}

MINIPAGE
\noindent\begin{minipage}{0.3\textwidth}% adapt widths of minipages to your needs
try 1
\end{minipage}%
\hspace{40mm}
\begin{minipage}{0.6\textwidth}
a) $f'(2)=$\\\

b) $f'(4)=$\\

c) $f'(6)=$\\

d) $f'(7)=$\\

e) $f'(8)=$
\end{minipage}
