\documentclass[12pt]{article}
\usepackage{graphicx}
\usepackage{amssymb}
\usepackage{epstopdf}
\usepackage{amsmath}
\usepackage{multicol}
\usepackage{tcolorbox}
\usepackage{geometry}
\usepackage{enumitem}
\usepackage{fancyhdr}

\DeclareGraphicsRule{.tif}{png}{.png}{`convert #1 `dirname #1`/`basename #1 .tif`.png}

\textwidth = 6.5 in
\textheight = 9 in
\oddsidemargin = 0.0 in
\evensidemargin = 0.0 in
\topmargin = -23pt
\headheight = 0.0 in
\headsep = 0.0 in
\parskip = 0.2in
\parindent = 0.0in
\pagestyle{fancy}
\pagenumbering{gobble}

\newtheorem{theorem}{Theorem}
\newtheorem{corollary}[theorem]{Corollary}
\newtheorem{definition}{Definition}
%\includegraphics [height=50mm, width=50mm]{PathInt.jpg}
\title{Title} 

\begin{document}

 Name:
 \begin{center}\large{6.2 Derivatives and Antiderivatives}\end{center}

\textbf{Finding anti-derivatives}\\
Let $f(x)=\cos x$. Find an antiderivative, $F(x)$, so that $F'(x)=\cos x$. Can you find more than one function?

\vfill
\begin{tcolorbox}
The \textit{indefinite integral} of $f$ is written as \underline{\hspace{8cm}}\\ %\int f(x)\,dx = F(x)+C,\\

where $F'(x)=f(x)$. Notice that there is a \underline{\hspace{4cm}} of antiderivative\\ 
functions.
\end{tcolorbox}

\begin{enumerate}
\item Evaluate the following antiderivatives.
	\begin{multicols}{2}
	\begin{enumerate}[itemsep=1cm]
	\item $\displaystyle \int 2x\,dx=$
	\vfill
	\item $\displaystyle \int x^2\,dx=$
	\vfill
	\item $\displaystyle \int x^{28}\,dx=$
	\vfill
	\item $\displaystyle \int 3\,dx=$
	\vfill
	\end{enumerate}
	\end{multicols}
	\vfill
\begin{tcolorbox}
\textbf{Power Rule for Integrals}\\

$\displaystyle \int x^n \, dx =$ \underline{\hspace{8cm}}
\end{tcolorbox}
	

\item Evaluate more antiderivatives.
	\begin{multicols}{2}
		\begin{enumerate}[itemsep=1cm]
	\item $\displaystyle \int 5e^x\,dx=$
	\vfill	
	\item $\displaystyle \int \frac{3}{t}\,dt=$
	\vfill
	\item $\displaystyle \int p^2(p+2)\,dp=$
	\vfill
	\item $\displaystyle \int \left[\frac{\theta+1}{\theta}+\sec^2 \theta \right]\,d\theta=$
	\vfill
	\item $\displaystyle \int e^{5t}\,dt=$
	\vfill
	\end{enumerate}
	\end{multicols}
	
\newpage

$\hspace{10px}$\\

\item Look at all of the antiderivatives you already know!

\begin{multicols}{2}
\begin{enumerate}[itemsep=1cm]
\item $\displaystyle \int \sin x\,dx=$
\item $\displaystyle \int \cos x\,dx=$
\item $\displaystyle \int \sec^2 x\,dx=$
\item $\displaystyle \int  x^7\,dx=$
\item $\displaystyle \int\frac{1}{x}\,dx=$
\item $\displaystyle \int e^x\,dx=$
\item $\displaystyle \int \frac{1}{1+x^2}\,dx=$
\item $\displaystyle \int \frac{1}{\sqrt{1-x^2}}\,dx=$
\end{enumerate}
\end{multicols}
\vfill
\begin{tcolorbox}
\textbf{Using Antiderivatives to Compute Definite Integrals}\\

$\displaystyle \int_0^3 f(x) \, dx =$ \underline{\hspace{4cm}}
\end{tcolorbox}
\vfill
\item Use an antiderivative to compute the following definite integrals


\begin{enumerate}
\item $\displaystyle \int_1^2 3x\,dx=$
\vfill
\item $\displaystyle \int_0^{\frac{\pi}{4}} \sin x\,dx=$
\vfill
\end{enumerate}

\item Find the area between the curves $y=x^2$ and $y=2-x^2$.
\vfill
\end{enumerate}
\end{document}
%%%%%%%%%%%%%%




