\documentclass[12pt]{article}
\usepackage{graphicx}
\usepackage{amssymb}
\usepackage{epstopdf}
\usepackage{amsmath}
\usepackage{multicol}
\usepackage{tcolorbox}
\usepackage{geometry}
\usepackage{enumitem}
\usepackage{fancyhdr}

\DeclareGraphicsRule{.tif}{png}{.png}{`convert #1 `dirname #1`/`basename #1 .tif`.png}

\textwidth = 6.5 in
\textheight = 9 in
\oddsidemargin = 0.0 in
\evensidemargin = 0.0 in
\topmargin = -23pt
\headheight = 0.0 in
\headsep = 0.0 in
\parskip = 0.2in
\parindent = 0.0in
\pagestyle{fancy}
\pagenumbering{gobble}

\newtheorem{theorem}{Theorem}
\newtheorem{corollary}[theorem]{Corollary}
\newtheorem{definition}{Definition}
%\includegraphics [height=50mm, width=50mm]{PathInt.jpg}
\title{Title} 

\begin{document}
%INSTRUCTOR NOTES
%No warm-up on quiz days. If wanted, today's warm-up would be solving equations.
 Name:
 \begin{center}\large{1.7 What's up with these functions?}\end{center}
\begin{enumerate}

\item All of these functions do something weird at $x=0$. Investigate the functions using your intuition, plugging in values, or using graphs. In particular, be prepared to share with the class:
	\begin{itemize}
	\item What is $f(0)$? Does it even exist?
	\item What happens to the function near $x=0$? Plug in nearby points or use a graph to support your reasoning.
	\end{itemize}
	
	\begin{enumerate}
	\item $\displaystyle f(x)=\frac{|x|}{x}$
	\vfill
	\item $\displaystyle g(x)=\frac{4x^2-5x}{x}$
	\vfill
	\item $\displaystyle h(x)=\frac{x^2+5}{x^2}$
	\vfill
	\item $\displaystyle k(x) = \begin{cases} 
      e^x& \text{if }x < 0 \\
      1-x & \text{if }x \geq 0 \\
   \end{cases}$
	\vfill
	\item $\displaystyle l(x)=\sin\left(\frac{1}{x}\right)$
	\vfill
	\end{enumerate}
\newpage
~
\begin{tcolorbox} 
\textbf{Definitions:}

\begin{itemize} 
\item A function $f$ is continuous at $c$ if $\displaystyle \lim_{x \to c} f(x) = f(c)$, which means ...

\vspace{30mm}

\item The limit $L$ of a function $f$ at $c$ exists if $\displaystyle \lim_{x \to c^+} f(x) =  \lim_{x \to c^-} f(x)$, which means...
 \vspace{25mm}

\end{itemize}
\end{tcolorbox}

\item The piecewise function $f(x)$ is graphed below.
	\begin{enumerate}
	\item For what value(s) of $x$ is this function not continuous?
	\item For what value(s) of $x$ does the limit not exist?
	\end{enumerate}
\vspace{5mm}
 \begin{tabular}{cl}  
         \begin{tabular}{c}
           \includegraphics[scale=.4]{1_8_Limitgraph2}
           \end{tabular}
           & \begin{tabular}{l}
             \parbox{0.5\linewidth}{%  change the parbox width as appropiate
           (c) Find the following:
	\begin{enumerate}
	\item $\displaystyle \lim_{x\to -3^+}f(x) = $
	\item $\displaystyle \lim_{x\to -3^-}f(x) = $
	\item $\displaystyle \lim_{x\to -3}f(x) = $
	\item $\displaystyle f(-3) = $
	\item $\displaystyle \lim_{x\to 2^+}f(x) = $
	\item $\displaystyle \lim_{x\to 2^-}f(x) = $
	\item $\displaystyle \lim_{x\to 2}f(x) = $
	\item $\displaystyle f(2) = $
	\end{enumerate}    }
         \end{tabular}  \\
\end{tabular}

\end{enumerate}

\begin{tcolorbox} 
\textbf{Intermediate Value Theorem:} Suppose $f$ is continuous on a closed interval $[a,b]$.\\ If $k$ is any number between $f(a)$ and $f(b)$, then there exists at least one number\\ $c$ in $[a,b]$ such that $f(c)=k$.
\end{tcolorbox}


\end{document}

\begin{enumerate}
\item For what value(s) of $x$ is this function not continuous?\\
For what value(s) of $x$ does the limit not exist?\\
\vspace{5mm}
 \begin{tabular}{cl}  
         \begin{tabular}{c}
           \includegraphics[scale=.4]{1_8_Limitgraph2}
           \end{tabular}
           & \begin{tabular}{l}
             \parbox{0.5\linewidth}{%  change the parbox width as appropiate
            Find the following:
	\begin{enumerate}
	\item $\displaystyle \lim_{x\to -3^+}f(x) = $
	\item $\displaystyle \lim_{x\to -3^-}f(x) = $
	\item $\displaystyle \lim_{x\to -3}f(x) = $
	\item $\displaystyle f(-3) = $
	\end{enumerate}    }
         \end{tabular}  \\
\end{tabular}

\item Find the value of $k$ that would make the function continuous.

	\[ g(x) = \begin{cases} 
      \frac{e^x-1}{x}& x\neq 0 \\
      k & x = 0 \\
   \end{cases} \]
\vfill
\item Find each limit. Include a table of values to illustrate your answer. \\

	\begin{multicols}{2}
	\begin{enumerate}[itemsep=3cm]
	\item $\displaystyle \lim_{x\to 0} \left(1+x\right)^{\frac{1}{x}}=$
	\item $\displaystyle \lim_{\theta \to 0} \frac{\sin(2\theta)}{\theta}=$
	\item $\displaystyle \lim_{y \to \infty} \frac{\sqrt{y^2+2}}{5y-6}=$
	\item $\displaystyle \lim_{t \to 1^+} \frac{|1-t|}{1-t}=$
\end{enumerate}
\end{multicols}


\item Find a value of $m$ that would make the limit exist. Find the limit.
$$ \lim_{x\to\infty}\frac{2x^3-6}{x^m+3}$$

\vfill
\item In each case sketch a graph with the given characteristics:
	\begin{enumerate}
	\item $f(4)$ is undefined and $\displaystyle \lim_{x\to4} f(x)=2$
	\vfill
	\item $f(3)=2$ and $\displaystyle \lim_{x\to3} f(x)$ does not exist
	\vfill
	\item $f(4) = 2$,  $\displaystyle \lim_{x\to4^-} f(x)=0$, and $\displaystyle \lim_{x\to4^+} f(x)=-\infty$
	\vfill
	\end{enumerate}

\end{enumerate}


\end{document} 

%%%%%%%%%
\begin{tcolorbox}
\textbf{Warm-up: } Solve the following equations for $t$.
\begin{multicols}{2}
\begin{enumerate}
\item $(t+1)^2=9$
\item $tx+x^2=5$
\end{enumerate}
\end{multicols}
\end{tcolorbox}

MINIPAGE
\noindent\begin{minipage}{0.3\textwidth}% adapt widths of minipages to your needs
try 1
\end{minipage}%
\hspace{40mm}
\begin{minipage}{0.6\textwidth}
a) $f'(2)=$\\\

b) $f'(4)=$\\

c) $f'(6)=$\\

d) $f'(7)=$\\

e) $f'(8)=$
\end{minipage}
